
\chapter{Test}
Successivamente in questo capitolo verranno descritti brevemente gli strumenti utilizzati per effettuare test sul codice sviluppato.\\[\baselineskip]\indent
Oltre a questi tools sono stati effettuati vari test con utenti durante le diverse fasi di implementazione. Basandosi sul bacino di utenti per il quale è stato pensato il sistema, sono state consultate per questi test persone interne all'università, anche iscritte ad altri corsi e quindi senza uno specifico background informatico.
Nelle prime fasi sono sono state richieste consulenze riguardo le scelte grafiche, presentando alcuni wireframes, e per quanto riguarda le specifiche a livello di regole di gioco ed usabilità. Successivamente sono stati effettuati test di accessibilità chiedendo di accedere al sistema da dispositivi diversi e testare le funzionalità basilari come registrazione, autenticazione e persistenza delle sessioni. Nelle fasi finali i test sono stati impostati come vere e proprie demo di scenari di utilizzo reali, giocando partite complete toccando tutte le funzionalità implementate.
\newpage

\section{Redux DevTools}
Per effettuare i test necessari a verificare che lo stato dell'applicazione fosse correttamente gestito sono stati usati gli strumenti offerti da Redux DevTools \cite{githubGitHubReduxjsreduxdevtools}.
Questi strumenti per sviluppatori permettono di migliorare il flusso di sviluppo di Redux o qualsiasi altra architettura che gestisca cambiamenti di stato. Può essere utilizzato come estensione per i principali browser, Chrome, Edge e Firefox, come app standalone oppure come componente React integrato nel client.

\begin{figure}[H]
\centering
\includegraphics[width=0.9\textwidth]{img/redux_devtools_usage.png}
\caption{Redux DevTools in uso}
\label{fig:reduxDevTools}
\end{figure}

Per lo sviluppo di questo progetto è stata utilizzata l'estensione messa a disposizione per Chrome, come mostrato nell figura \ref{fig:reduxDevTools}, nella quale è possibile vedere lo stato in tempo reale di Redux.

\begin{figure}[H]
\centering
\includegraphics[width=0.3\textwidth]{img/logos/redux_devtool.jpg}
\caption{Redux DevTools logo}
\label{fig:reduxDevToolsLogo}
\end{figure}


\section{Axe DevTools}

Per testare aspetti di accessibilità sono stati invece utilizzati gli strumenti messi a disposizione da axe DevTools \cite{dequeDevToolsDeveloper}.
Questi tools permettono, sempre attraverso un'estensione disponibile per il browser Chrome, di analizzare vari aspetti di accessibilità relativi ad una specifica pagina web o ad una sua sottoparte. 

L'utilizzo dei tools è visibile nella figura \ref{fig:axeDevTools}, nella quale si può inoltre osservare la struttura di suddivisione utilizzata per attribuire un livello di gravità al problema riscontrato.


\begin{figure}[H]
\centering
\includegraphics[width=\textwidth]{img/axe_devtools_usage.png}
\caption{Axe DevTools in uso}
\label{fig:axeDevTools}
\end{figure}

Il set di tool di cui ci siamo serviti per monitorare ed eventualmente correggere questioni riguardanti l'accessibilità nel presente applicativo. Axe DevTools è un insieme di strumenti leader del settore, che permette tramite un'estensione installabile direttamente su browser di ispezionare una pagina web o sottoparti di essa allo scopo di individuare problemi di accessibilità. Il tool poi procede con il suddividere i problemi in varie categorie a seconda della gravità dei problemi eventualmente riscontrati:

\begin{figure}[H]
\centering
\includegraphics[width=0.4\textwidth]{img/logos/axeDevtools_logo.png}
\caption{Axe DevTools logo}
\label{fig:axeDevToolsLogo}
\end{figure}


\section{Lighthouse}
Lighthouse è uno strumento automatizzato completamente open-source pensato per migliorare le prestazioni, la qualità e la correttezza delle applicazioni Web \cite{githubLighthouse}.

Durante il controllo di una pagina, questo tool esegue una serie di test sulla pagina e quindi genera un rapporto sul rendimento della pagina, visibile nella figura \ref{fig:lighthouseReport}. Da qui è possibile utilizzare eventuali test falliti come indicatori di quali azioni si possano intraprendere per migliorare la propria applicazione.

\begin{figure}[H]
\centering
\includegraphics[width=0.8\textwidth]{img/lighthouse_usage.png}
\caption{Lighthouse in uso}
\label{fig:lighthouseReport}
\end{figure}

\begin{figure}[H]
\centering
\includegraphics[width=0.3\textwidth]{img/logos/lighthouse_logo.png}
\caption{Lighthouse logo}
\label{fig:lighthouseLogo}
\end{figure}
