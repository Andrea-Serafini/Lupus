\chapter{Conclusioni}

\section{Note finali}
Lo sviluppo di questo progetto ha permesso di approfondire, e familiarizzare con, i framework ed i tool utilizzati, oltre che apprendere attraverso tentativi ed errori quali fossero le metodologie migliori per impostare un lavoro di questo tipo.

Il risultato finale ottenuto rispetta tutti i requisiti fondamentali definiti in fase di analisi e proposta di progetto, oltre che tutte le aspettative dal punto di vista di usabilità e funzionalità del sistema. Dopo aver effettuato test con amici e colleghi non solo il livello di implementazione è stato ritenuto più che soddisfacente, ma anche l'idea alla base della digitalizzazione del gioco ha riscosso molto successo, con numerose richieste per una futura release così da poter continuare ad utilizzarlo.

I problemi riscontrati durante lo sviluppo hanno a volte rappresentato sfide importanti, soprattutto non avendo la possibilità di confrontarsi con un gruppo, ma per ognuno è stata trovata una soluzione o una via alternativa che nel complesso ho sempre ritenuto all'altezza delle personali aspettative.

\section{Sviluppi futuri}
In futuro ampliando ed approfondendo il progetto i punti salienti che presentano possibilità di ulteriori sviluppi sono risultati:

\begin{itemize}
    \item \textbf{Applicazione mobile:} come era stato ipotizzato in fase di definizione delle specifiche di progetto sarebbe interessante, oltre che utile per migliorarne la fruibilità, sviluppare un'applicazione mobile che faccia uso di librerie specifiche.
    \item \textbf{Comunicazione interna:} attualmente il sistema prevede che per apprezzare al meglio il gioco implementato gli utenti si ritrovino nello stesso luogo, o che utilizzino sistemi alternativi di comunicazione, come ad esempio Discord. In futuro sfruttando le capacità di PeerJS sarebbe interessante aggiungere la possibilità di comunicare all'interno della lobby, testualmente, con un canale audio, o addirittura video.
\end{itemize}