\begin{abstract}
Questo progetto nasce con l'obiettivo di creare una versione digitale del gioco Lupus, conosciuto anche come Mafia \cite{mafiaWikipedia}, un gioco conosciuto ormai da tutti, molto popolare tra i gruppi di universitari.\\[\baselineskip]\indent
Il gioco prevede che ci siano almeno 6 giocatori ed un narratore esterno che orchestri la partita. Prima di iniziare la partita, il narratore assegna casualmente un ruolo a ciascun giocatore. Generalmente questa assegnazione viene effettuata con l'ausilio di carte da gioco o token fisici, poiché è fondamentale per lo svolgimento che i ruoli rimangano segreti per tutta la partita. I ruoli presenti nel gioco possono variare a seconda del numero di giocatori o del tipo di partita che si vuole effettuare.\\[\baselineskip]\indent
Il traguardo che si pone questo progetto è quello di rendere disponibile un sistema che permetta ad un gruppo di amici di coordinare la partita senza bisogno di avere un narratore presente, e che dia al tempo stesso la possibilità di giocare sia in presenza, sia da remoto.
Il sistema andrà ad offrire le funzionalità base di autenticazione per consentire l'identificazione di un giocatore in maniera univoca, mantenendo lo storico delle partite giocate e di alcune statistiche relative ai risultati ottenuti.\\[\baselineskip]\indent
Durante lo sviluppo del progetto si sfrutteranno ed approfondiranno varie tecnologie, meglio descritte in seguito, affrontate nell'ambito del corso di \emph{Applicazioni e Servizi Web}. Sarà quindi necessario uno studio delle \emph{best practices} attuali, sfruttando le documentazioni ufficiali e risorse web correlate.\\[\baselineskip]\indent
La seguente relazione tratterà il processo di sviluppo del sistema descrivendo prima i requisiti e gli obiettivi del progetto, per poi scendere nel dettaglio delle scelte di design effettuate per le sue componenti. Verrà poi fatta una panoramica sulle tecnologie scelte ed utilizzate, seguita da alcuni estratti di codice che ne mostrano l'applicazione. L'ultima parte riguarderà infine eventuali test e le indicazioni per il \emph{deployment} degli artefatti.

\end{abstract}