\begin{abstract}
Questo progetto nasce con l'obiettivo di creare una versione digitale del gioco Lupus, conosciuto anche come Mafia \cite{mafiaWikipedia}, un gioco conosciuto ormai da tutti, molto popolare tra i gruppi di universitari.\\[\baselineskip]\indent
Il gioco prevede che ci siano almeno 6 giocatori e un narratore esterno che orchestri la partita. Prima di iniziare la partita, il narratore assegna casualmente un ruolo a ciascun giocatore. Generalmente questa assegnazione viene effettuata con l'ausilio di carte da gioco o token fisici, poiché è fondamentale per lo svolgimento che i ruoli rimangano segreti per tutta la partita. I ruoli presenti nel gioco possono variare a seconda del numero di giocatori o del tipo di partita che si vuole effettuare.\\[\baselineskip]\indent
Durante lo svolgimento del gioco ci sarà un alternarsi di fasi diurne e notturne nelle quali i diversi personaggi potranno intraprendere azioni specifiche. Il narratore avrà il compito di supervisionare l'alternarsi delle fasi di gioco, tenere traccia delle decisioni dei giocatori e delle loro azioni.\\[\baselineskip]\indent
Nella versione originale e più semplice del gioco i ruoli sono quelli di \emph{cittadino} e \emph{mafioso}, per cui si andranno a creare due squadre con obiettivi contrapposti, identificare i mafiosi e uccidere i cittadini. A ogni turno di notte il narratore dovrà chiedere ai mafiosi chi vogliono uccidere all'insaputa dei cittadini, mentre durante il turno di giorno tutti i giocatori dovranno decidere chi accusare di essere un mafioso, consapevoli però che i mafiosi cercheranno di influenzare a loro favore la decisione. \\[\baselineskip]\indent
La partita termina nel momento in cui tutti i giocatori di una delle due squadre hanno la meglio sugli altri, avendo ucciso tutti i cittadini o imprigionato tutti i mafiosi. A questo punto sarà sufficiente distribuire nuovamente i ruoli per iniziare una nuova partita.
\end{abstract}